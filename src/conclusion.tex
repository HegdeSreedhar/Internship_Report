\section{Conclusion}
With the shift towards using Automotive Ethernet as a backbone for communication within the vehicles, the need for a reliable middleware like SOME/IP becomes so much more vital. The automotive software developed using service oriented architecture provides flexibility and modularity and also can be integrated with the existing software with ease. To support these types of architecture, a communication protocol named SOME/IP was developed. This work focused on the realization of the SOME/IP technology and develop applications based on the service oriented architecture. 
\par Overall the results obtained in this activity met all of the objectives set when planning it. During the initial planning phase, the intention was to setup a demonstrator that can be used to understand the SOME/IP technology. However during the concept development phase, it was evident that there can be several issues that can occur commonly and there is a need to document the solutions for these issues for future use cases. Therefore, a troubleshooting guide was created through which other developers can be benefited by referring to it and save time whenever few commonly known errors appear during the development and testing of the SOME/IP based applications. Also, POSIX based systems and service oriented architectures are relatively new in the automotive domain, this experiment can further help to build more complex applications as the basic prerequisites are well documented based on the outcome of this project. With the setup of the vsomeip stack on multiple consumer based SBCs, it serves as a reference point to further develop and execute SOME/IP based applications on automotive specific controllers for real-time industry projects.

\section{Future scope}
Within this project, mostly UDP is used as the transport layer to send and receive the message packets. To send and receive message packets, UDP is mostly used as the transport layer in this project. TCP-based communication can also be implemented for more reliable communication. Messages can be transferred via TCP or UDP using the vsomeip stack. Additional tests can be performed to compare the outcomes of UDP-based communication. Furthermore, runtime analysis of SOME/IP applications can be performed to determine the resource usage by each service. Additionally, the maximum throughput can be tested by significantly increasing the number of requests to a specific service from various clients. Because the requirements in actual real-time projects are usually very demanding, this type of information can be extremely useful. Taking these aspects into account, additional future works may use this demonstrator as a starting point for more in-depth analysis.
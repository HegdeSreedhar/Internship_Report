% just insert your text here

Communication is essential in modern vehicles to establish a link between the ECUs in the network. In addition, as the number of ECUs and high-performance controllers grows, so does the need for more bandwidth than traditional in-vehicle networks such as CAN, Flexray, and MOST can provide. With the introduction of Ethernet into the automotive domain, bandwidths of up to 1 Gb/s can now be achieved within the vehicle network. The use of Ethernet benefits systems such as ADAS and infotainment significantly. However, in order to transmit and receive data at a significantly high data rate, a robust communication control mechanism is required. The use of Ethernet benefits systems such as ADAS and infotainment substantially. However, in order to transmit and receive data at a remarkably high data rate, a robust communication control mechanism is required. With the growing interest in POSIX-based systems in the automotive domain, service oriented architecture (SOA) plays an important role in meeting the needs of technology-driven applications. The core of SOA is remote procedure calling (RPC) and the Client-Server mechanism. To realize these concepts, there is a need for a middleware that is specifically designed to run automotive applications smoothly. To accomplish this, SOME/IP middleware was introduced in the automotive context. As more applications migrate to Adaptive AUTOSAR, SOME/IP is well suited to serve as a communication control protocol alongside existing communication technologies.

In this report, a detailed study of the SOME/IP technology is conducted. In order to understand the working of SOME/IP technology, the open source library vsomeip offered by GENIVI is used. A demonstrator consisting of target hardware running with different underlying architectures such as x64, armv7 and armv8 is setup. The devices are connected with each other on a network using Ethernet. The working is realized by running applications based on vsomeip stack on theses hardware devices. Also, a troubleshooting guide consisting of the commonly faced issues and faults while using the SOME/IP technology have been documented as a reference document.

% just insert your text here

Communication is essential in modern vehicles to establish a link between the ECUs in the network. In addition, as the number of ECUs and high-performance controllers grows, so does the need for more bandwidth than traditional in-vehicle networks such as CAN, Flexray, and MOST can provide. With the introduction of Ethernet into the automotive domain, bandwidths of up to 1 Gb/s can now be achieved within the vehicle network. The use of Ethernet benefits systems such as ADAS and infotainment significantly. However, in order to transmit and receive data at a significantly high data rate, a robust communication control mechanism is required. The use of Ethernet benefits systems such as ADAS and infotainment substantially. However, in order to transmit and receive data at a remarkably high data rate, a robust communication control mechanism is required. With the growing interest in POSIX-based systems in the automotive domain, service oriented architecture (SOA) plays an important role in meeting the needs of technology-driven applications. The core of SOA is remote procedure calling (RPC) and the Client-Server mechanism. To realize these concepts, there is a need for a middleware that is specifically designed to run automotive applications smoothly. To accomplish this, SOME/IP middleware was introduced in the automotive context. As more applications migrate to Adaptive AUTOSAR, SOME/IP is well suited to serve as a communication control protocol alongside existing communication technologies.

This report conducts a thorough examination of the SOME/IP technology. In Chapter 2, the details of the current systems and the most recent technology are explained. The open source library vsomeip provided by GENIVI is used to understand the functioning of SOME/IP technology. A demonstrator is set up, consisting of target hardware running on various underlying architectures such as x64, armv7, and armv8. The devices are linked together on a network via Ethernet. The functionality is achieved by running applications based on the vsomeip stack on these hardware devices. In Chapter 3, these concepts and their implementation are explained in detail. The fourth chapter focuses on the implementation's outcomes.The fourth chapter focuses on the outcomes of the implementation. As a reference document, a troubleshooting guide containing the most commonly encountered issues and faults when using SOME/IP technology has also been documented. Chapter 5 provides a conclusion to this report as well as the future scope of the work.